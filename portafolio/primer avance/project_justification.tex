\documentclass[conference]{IEEEtran}
\IEEEoverridecommandlockouts
% The preceding line is only needed to identify funding in the first footnote. If that is unneeded, please comment it out.
\usepackage{cite}
\usepackage{amsmath,amssymb,amsfonts}
\usepackage{algorithmic}
\usepackage{graphicx}
\usepackage{textcomp}
\usepackage{xcolor}
\def\BibTeX{{\rm B\kern-.05em{\sc i\kern-.025em b}\kern-.08em
    T\kern-.1667em\lower.7ex\hbox{E}\kern-.125emX}}
\begin{document}

\title{PROJECT JUSTIFICATION}

\author{
\IEEEauthorblockN{Andrea Michel Quintero Montalvo}
\IEEEauthorblockA{\textit{Universidad Tecnológica de Tijuana}\\
Tijuana, México \\
0318121702@ut-tijuana.edu.mx}
\and
\IEEEauthorblockN{Bryan Santiago Mora Molina}
\IEEEauthorblockA{\textit{Universidad Tecnológica de Tijuana}\\
Tijuana, Mexico \\
0320128617@ut-tijuana.edu.mx}
\and
\IEEEauthorblockN{María Del Carmen González García}
\IEEEauthorblockA{\textit{Universidad Tecnológica de Tijuana}\\
Tijuana, México \\
0320129810@ut-tijuana.edu.mx}
\and
\IEEEauthorblockN{Ricardo Alfredo Nieto Padilla}
\IEEEauthorblockA{\textit{Universidad Tecnológica de Tijuana}\\
Tijuana, México \\
0319124527@ut-tijuana.edu.mx}
\and
\IEEEauthorblockN{Roberto Ortiz Monroy}
\IEEEauthorblockA{\textit{Universidad Tecnológica de Tijuana}\\
Tijuana, México \\
0320127817@ut-tijuana.edu.mx}
}

\maketitle

\begin{abstract}
This document justifies the development of a Progressive Web App (PWA) for Welcomex, highlighting the advantages and necessity of leveraging this technology to enhance accessibility, user experience, and data management.
\end{abstract}

\section{Problem \& Solution}
In response to the pressing challenges posed by migration in Mexico, this project introduces Welcomex, a dedicated web platform designed for the registration and analysis of migration statistics. With a focus on providing a specific and efficient tool, Welcomex aims to collect pertinent data and conduct a comprehensive analysis of the migration landscape in the country. By facilitating communication, offering reassurance, and visualizing migration data, Welcomex strives to address the need for a more effective understanding and management of migration phenomena. Motivated by the desire to enhance decision-making, support the migrant population, and contribute to research efforts, Welcomex emerges as a comprehensive tool fostering continuous monitoring, transparency, and alignment with national and international migration objectives. The project addresses the complexities migrants face, offering insights that can inform government policies, urban planning, and regional development to anticipate and address demographic changes and specific needs in high migration areas.

\section{Project justification}
The decision to implement a Progressive Web App (PWA) for Welcomex is grounded in strategic and economic considerations that align with the project's objectives. This justification highlights several key factors that make a PWA the preferred choice over a native application.
\subsection{Existing Web Application Functionality}
Welcomex benefits from an already robust web application meticulously designed for the proficient management of migrant resources. The current iteration exhibits a seamless user experience, encompassing vital functionalities tailored to meet the unique needs of clients. As the transition to a Progressive Web App (PWA) is contemplated, it becomes evident that the essential functionalities predominantly involve harnessing the capabilities of the device's camera and microphone. Given the web application's existing prowess in addressing user requirements, adapting it into a PWA is a logical progression that capitalizes on its current functionality while embracing the advantages inherent to the PWA framework. This approach ensures a smooth migration to the PWA paradigm without sacrificing the efficiency and effectiveness that users have come to expect from the existing web application.
\subsection{Inherent Multiplatform Nature}
The decision to opt for a PWA is underpinned by the intrinsic multiplatform capabilities it brings to Welcomex. Unlike native applications that require tailored development for each operating system, PWAs seamlessly operate across diverse devices and platforms. This inherent adaptability is instrumental in streamlining the development process, as a singular PWA codebase suffices for deployment on various devices, whether running on iOS, Android, or other operating systems. The multiplatform nature not only expedites the development timeline but also significantly mitigates the economic resources traditionally expended on platform-specific development efforts. Welcomex, by embracing the inherent versatility of PWAs, ensures that its migration statistics platform is universally accessible, fostering inclusivity and usability across a spectrum of devices and operating systems.
\subsection{Instant and Automatic Updates}
A crucial advantage of opting for a PWA is the ease of instant and automatic updates. This ensures that users always have access to the latest version without manual interventions. The consistent and updated experience enhances the efficiency and security of the application.
\subsection{Avoidance of Traditional App Distribution Limitations}
By choosing a PWA, the project aims to sidestep the limitations and costs associated with traditional app distribution platforms. Eliminating the need for users to download the application from a store removes intermediaries and potential expenses tied to transactions on these platforms. This strategy not only simplifies the implementation process but also reduces long-term operational costs, allowing for a more efficient distribution of the application among the migrant audience.
\subsection{Offline Functionality and Performance}
PWAs offer offline functionality, allowing users to access certain features even without an active internet connection. This is particularly valuable in scenarios where the user may face connectivity challenges. Additionally, PWAs are known for their efficient performance, providing a smooth user experience even in low-bandwidth conditions.
\subsection{Cost-Efficiency in Development and Maintenance}
A pivotal factor motivating the selection of a Progressive Web App (PWA) for Welcomex is the substantial cost-efficiency realized in both development and maintenance endeavors. The streamlined development process associated with PWAs, leveraging a single codebase for multiplatform deployment, significantly reduces the resources traditionally allocated to bespoke development for various operating systems.

Moreover, the automatic and instantaneous updates facilitated by PWAs eliminate the need for manual interventions, ensuring that users consistently access the latest features and security enhancements. This not only bolsters the efficiency and relevance of the application but also diminishes the financial burden associated with manual update processes.

By sidestepping the conventional expenses linked to traditional app distribution channels, such as app store transactions and intermediaries, Welcomex embraces a cost-effective strategy. The decision to forgo the necessity for users to download the application from a store not only simplifies the implementation process but also results in long-term operational cost reductions, enabling a more resource-efficient distribution of the application among the migrant audience.

In essence, the adoption of a PWA aligns seamlessly with Welcomex's commitment to optimizing resource allocation, ensuring a sustainable and economically viable solution for the development, deployment, and maintenance of the migration statistics platform.
\subsection{Scalability and Adaptability}
Welcomex recognizes the dynamic nature of technology landscapes and anticipates future advancements and evolving project needs. The choice to implement a Progressive Web App (PWA) is rooted in its inherent scalability and adaptability, providing a foundation that can effortlessly accommodate growth and changes in technology.

As Welcomex evolves, introducing new features and functionalities becomes a streamlined process within the PWA framework. The modular nature of PWAs facilitates seamless integration of updates, ensuring that the application remains agile and responsive to emerging requirements. This adaptability is particularly crucial in the context of a project like Welcomex, where the landscape of migration statistics and management may evolve, necessitating adjustments and expansions.

The scalability of PWAs allows Welcomex to efficiently scale its operations, handle increased user loads, and incorporate enhancements without undergoing extensive redevelopment efforts. By choosing a PWA, Welcomex positions itself to meet the future demands of its user base while maintaining a high level of performance, responsiveness, and user satisfaction. The commitment to scalability and adaptability underscores Welcomex's forward-thinking approach in addressing the fluid dynamics of migration management.
\subsection{Compliance with Open Web Standards}
Embracing open web standards is a strategic imperative for Welcomex, and the decision to implement a Progressive Web App (PWA) aligns seamlessly with this commitment. PWAs adhere to universally recognized and accepted web standards, ensuring interoperability and compatibility across a diverse range of browsers.

By adhering to open standards, Welcomex guarantees that its migration statistics platform is accessible to the broadest possible audience, irrespective of the browser preferences of its users. This inclusivity is vital in catering to the diverse technological landscape that migrants may encounter. It mitigates barriers to access and ensures a consistent user experience regardless of the browser or device employed, fostering an environment of accessibility and usability for all stakeholders involved.

Furthermore, compliance with open web standards reinforces Welcomex's dedication to transparency and collaborative development. It aligns the project with global best practices and facilitates ongoing improvements, updates, and collaborations within the broader web development community. The commitment to open standards not only ensures the longevity and sustainability of the Welcomex platform but also signifies a pledge to foster an open and accessible digital ecosystem for the benefit of all users.
\section{Conclusion}
In short, Welcomex emerges as a comprehensive solution to Mexico's migratory challenges, introducing a dedicated web platform for the registration and analysis of migration statistics. The strategic choice of a Progressive Web App (PWA) is grounded in the proven effectiveness of the current web application, adapting it to leverage the advantages of PWAs. This decision ensures universal accessibility, automatic updates, economic efficiency, offline functionality, and efficient performance. The inherent scalability of PWAs positions Welcomex for future advancements, while adherence to open standards guarantees accessibility and transparency. Therefore, the adoption of a PWA reflects Welcomex's commitment to resource optimization and provides a sustainable solution for the development, implementation, and maintenance of its migration statistics platform.
\end{document}
