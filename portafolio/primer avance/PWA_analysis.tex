\documentclass[conference]{IEEEtran}
\IEEEoverridecommandlockouts
\usepackage{cite}
\usepackage{amsmath,amssymb,amsfonts}
\usepackage{algorithmic}
\usepackage{graphicx}
\usepackage{textcomp}
\usepackage{xcolor}
\def\BibTeX{{\rm B\kern-.05em{\sc i\kern-.025em b}\kern-.08em
    T\kern-.1667em\lower.7ex\hbox{E}\kern-.125emX}}
\begin{document}

\title{A Comprehensive Analysis of Progressive Web Applications}

\author{\IEEEauthorblockN{Andrea Quintero}
\IEEEauthorblockA{\textit{Progressive Web Apps} \\
\textit{Universidad Tecnológica de Tijuana}\\
Tijuana, México}
}

\maketitle

\section*{Introduction}
The landscape of application development has undergone many changes in recent years, resulting in a wide range of solutions to meet the dynamic needs of users. This paper explores the key concepts and characteristics that define PWAs, delving into their unique features and comparing them with traditional web applications and native counterparts. As our dependency on digital interfaces continues to grow, understanding those application types becomes increasingly important for developers.
\section{Progressive Web Applications}
A Progressive Web Application (PWA) is an application or software that was built using web platform technologies but working as a specific-platform application.
In other words, a Progressive Web App is an application that can operate on various devices and platforms, functioning both as a web app and being installable on a device as a native app.

PWAs embody the principle of Progressive Enhancement, a design philosophy that ensures compatibility across devices, ranging from older browsers to the latest. This adaptability is crucial in an era where diverse devices with different capabilities coexist. 

Responsive Design is another fundamental aspect of PWA, enabling them to adjust to diverse screen sizes on different devices, such as smartphones, tablets, or desktop computers. This flexible design ensures that the PWA maintains its usability and visual attractiveness across a range of devices, providing a cohesive experience to users.

PWAs aim to provide users with an App-Like Experience, and the introduction of Service Workers plays a crucial role in achieving this goal. These background scripts enable advanced features such as push notifications and offline functionality, elevating the overall user experience to a level comparable to native applications.

Progressive Web Apps (PWAs) are changing how we use digital applications. They blend the best parts of regular websites and mobile apps, giving users a flexible and user-friendly experience. With features like adaptability, working offline, and strong security, PWAs are a great choice in today's world where people want smooth and feature-packed apps.
\section{Web Applications}
A web application is software designed to be executed in a web-based environment. Some of its most common applications include ecommerce, instant messaging and social networking, online banking, among others.

Advantages of implementing web applications include their accessibility from any device (laptop, smartphone, tablet, etc.) at any time and without the need for downloading or installation. Additionally, they simplify the development process and require fewer resources due to a single code base. This is in contrast to native apps, where a separate app is needed for each operating system. Furthermore, web applications can easily scale to accommodate a growing number of users by upgrading server infrastructure, making them suitable for businesses with fluctuating user bases.

Web applications require a stable internet connection for optimal performance. While some apps offer limited offline functionality, they generally rely on an internet connection for full functionality, which can be a limitation in areas with poor connectivity. Additionally, these types of applications are susceptible to security vulnerabilities, such as cross-site scripting and SQL injection, which need to be addressed through proper security measures.

In brief, web applications offer unparalleled accessibility and scalability but depend on a stable internet connection. Their advantages in streamlined development must be balanced with awareness of security vulnerabilities, emphasizing the need for robust protective measures.

\section{Native Applications}
A native application is tailored for a specific platform, like iOS or Android, maximizing the utilization of the device's hardware and software capabilities. Common applications cover a wide range of categories, including games, social media, productivity tools, banking, and more.

Native apps generally offer better performance compared to web apps, as they are optimized for the specific platform and can utilize native APIs. Moreover, they can be conveniently distributed through official app stores, such as the Apple App Store and Google Play, offering users a centralized and secure channel. Additionally, native apps can incorporate platform-specific security features and adhere to platform guidelines, minimizing the risk of potential vulnerabilities.

In the other hand, developing and maintaining separate codebases for different platforms can be more complex and time-consuming compared to cross-platform or web development., resulting in higher development costs compared to building a single cross-platform or web application. Additionally, native apps must go through an approval process on app stores, which can lead to delays and potential rejection based on platform guidelines.

In summary, native applications offer high performance, access to device features, and a platform-specific user experience but may involve higher development costs and complexities compared to web applications. The choice between web and native applications often depends on the specific needs, goals, and resources of the development project.

 \section{Conclusion}
In conclusion, the choice between Progressive Web Applications, Web Applications, and Native Applications depends on various factors such as performance requirements, development complexity, and target audience. Progressive Web Applications offer a balance between accessibility and advanced features, making them suitable for a wide range of scenarios. Web Applications excel in accessibility and scalability but require a stable internet connection. Native Applications provide high performance and platform-specific experiences but involve higher development costs. Understanding the strengths and limitations of each approach is crucial for developers in selecting the most suitable solution for their projects.
\begin{thebibliography}{00}
\bibitem{b1} Progressive web apps. MDN Web Docs. (n.d.). https://developer.mozilla.org/en-US/docs/Web/Progressive\_web\_apps
\bibitem{b2} Roca, J. (2020, September 17). What is Cross Platform Software?. Triangle. https://www.triangle.es/en/what-is-cross-platform-software/
\bibitem{b3} Almeida, J. (2023, December 21). Web app development: What you need to know. DistantJob. https://distantjob.com/blog/web-app-development/
\bibitem{b4} What are the advantages and disadvantages of web applications?. iTrobes. (2023, August 3). https://www.itrobes.com/what-are-the-advantages-and-disadvantages-of-web-applications/ 
\bibitem{b5} What is a native app? (Beginner’s Guide + Examples). Mighty Networks. (2023, August 28). https://www.mightynetworks.com/resources/native-app
\bibitem{b6} What’s the Difference Between Web Apps, Native Apps, and Hybrid Apps? Amazon AWS. (1981). https://aws.amazon.com/compare/the-difference-between-web-apps-native-apps-and-hybrid-apps/
\end{thebibliography}
\end{document}
